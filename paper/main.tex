\documentclass[utf8,german,analyse,lot,lof,lol]{zihpub}

\usepackage{caption}
\usepackage{datetime}
\usepackage{etoolbox}
\usepackage[acronyms,nonumberlist,nopostdot,toc,savewrites,xindy]{glossaries}
\usepackage[section,newfloat]{minted}
\usepackage[super]{nth}
\usepackage{pgfplots}
\usepackage{siunitx}
\usepackage{tcolorbox}

\makeglossaries

\newacronym{ocr}{OCR}{\textit{optical character recognition}}


\newenvironment{code}{\captionsetup{type=listing}}{}
\SetupFloatingEnvironment{listing}{name=Quelltext}

\BeforeBeginEnvironment{minted}{\begin{tcolorbox}}
\AfterEndEnvironment{minted}{\end{tcolorbox}}

\newcommand{\lstfont}[1]{\color{#1}\small\ttfamily}

\definecolor{keyword-green}{RGB}{0, 128, 0}

\author{Jan Stephan}
\title{Texterkennung in topographischen Karten: Untersuchungen mit dem Deep-Learning-Framework Keras in einer
       HPC-Systemumgebung}
\matno{3755136}
\betreuer{Dr.\ Peter Winkler}
\date{28. September 2018}
% \copyrighterklaerung{Die nachfolgend aufgeführten Wortmarken sind Markenzeichen ihrer jeweiligen Inhaber, auch wenn sie
%                     im Text dieser Arbeit nicht als solche gekennzeichnet sind. CUDA{\textregistered},
%                     Quadro{\textregistered} und GeForce{\textregistered} sind eingetragene Markenzeichen der Firma
%                     NVIDIA. DirectX{\textregistered} ist ein eingetragenes Markenzeichen der Firma Microsoft.
%                     OpenGL{\texttrademark} und OpenCL{\texttrademark} sind nichteingetragene Markenzeichen der
%                     Khronos Group. Xeon Phi{\texttrademark} ist ein nichteingetragenes Markenzeichen der Firma Intel.}
\acknowledgments{Für die tatkräftige Hilfe bei der manuellen Extraktion der Realdaten aus dem Kartenmaterial bedanke ich
                 mich herzlich bei N. Werner. TODO: Vornamen erfragen}

\bibfiles{bibliography.bib}
\def\UrlBreaks{\do\/\do-}

% Germanisiere LaTeX
\deftranslation[to=German]{Acronyms}{Abkürzungsverzeichnis}
\deftranslation[to=German]{Glossary}{Glossar}
\renewcommand*{\listlistingname}{Quelltextverzeichnis}

\fancyhf{}
\fancyhead[L]{Jan Stephan -- Texterkennung in topographischen Karten}

\begin{document}

\glsaddall

\printglossary
\printglossary[type=\acronymtype]
%\printglossary[type=nomencl,style=nomenclstyle]

\section{Einleitung}
\label{einleitung}

\subsection{Motivation}
\label{einleitung:motivation}

Die Analyse historischer topographischer Karten ist eine bedeutende Aufgabe der Raum- und Landschaftsplanung, um die
historische Entwicklung von Siedlungen und Naturräumen nachvollziehen zu können. Besondere Aufmerksamkeit kommt dabei
der automatisierten Texterkennung zu, um digitalisiertes Kartenmaterial schnell katalogisieren, sortieren oder
durchsuchen zu können.

Die kombinierte Texterkennung in und -extraktion aus Bildern ist ein Standardproblem der Bildanalyse sowohl im Bereich
des klassischen \textit{Computer Vision} (d.h. Objekterkennung durch spezielle Algorithmen) als auch des seit einigen
Jahren populärer gewordenen Gebiet des maschinellen Lernens (siehe auch Abschnitt~\ref{einleitung:forschung}). Dieser
\gls{ocr} genannte Prozess kann in zahlreichen Anwendungsdomänen eingesetzt werden, wie etwa der automatisierten
Erkennung von Kennzeichen, der Digitalisierung von Büchern oder der Informationsextraktion aus nicht digitalisierten
Verträgen oder Business-Dokumenten.

Alle beispielhaft aufgeführten Fälle haben jedoch gemeinsam, dass die Texterkennung aufgrund des starken Kontrastes
zwischen Text und Hintergrund und des Mangels an störenden Artefakten relativ einfach möglich ist. Die Anwendung auf
historische topographische Karten gestaltet sich aufgrund einiger Faktoren vergleichsweise schwierig. So ist das
Farbspektrum trotz der Fülle an Informationen in der Regel auf die Farbe des (vergilbten) Papiers und eine einzige
Tintenfarbe eingeschränkt. Das führt dazu, dass der zu extrahierende Text sich farblich nicht von weiteren Markierungen
auf der Karte, wie etwa Straßen, Bäumen oder Höhenzügen, unterscheiden lässt und somit zahlreiche störende Artefakte
bei der Erkennung ignoriert werden müssen. Darüber hinaus ist die Linienstärke der Buchstaben historischer Schriftarten
stellenweise sehr dünn, sodass dickere Linien des Bildhintergrundes weitaus dominanter erscheinen als der eigentlich
vordergründige Text.

Diese Schwierigkeiten machen die Texterkennung in historischen topographischen Karten zu einer besonders
herausfordernden und interessanten Aufgabe im Bereich des maschinellen Lernens.

\subsection{Forschungsstand}
\label{einleitung:forschung}

Die Idee der maschinellen \gls{ocr} beschäftigt Ingenieure, Informatiker und Erfinder seit mehr als einem Jahrhundert.
Bereits 1914 stellte Edmund Fournier d'Albe ein \textit{Optophon} vor, das nach Erkennung eines Buchstabens einen
bestimmten Ton abspielte und auf diese Weise Blinden das Lesen ermöglichen sollte (vgl.~\cite{albe1914}). Wenige Jahre
später meldete der in Dresden bei der Firma \textit{Zeiss Ikon} tätige Emanuel Goldberg ein Patent für die von ihm
entwickelte \glqq statistische Maschine\grqq\ an, mit deren Hilfe die in Form von Buchstaben- und Ziffernkombinationen
vorliegenden Metadaten auf Filmrollen durchsucht werden konnten (vgl.~\cite{goldberg1931}).

In jüngerer Zeit ist die \gls{ocr} im Bereich des maschinellen Lernens häufiger untersucht worden. Eines der
bekanntesten Projekte in diesem Bereich ist die zunächst von der Firma \textit{Hewlett-Packard}, dann von der Firma
\textit{Google} entwickelte \textit{Tesseract OCR Engine} (vgl.~\cite{smith2007}), die mittlerweile neben
algorithmusbasierter Texterkennung den Einsatz eines neuronalen Netzes ermöglicht (vgl.~\cite{tesseract40}).

Jaderberg et al.\ präsentierten 2016 ein mehrstufiges Verfahren für die Texterkennung in Fotografien. Bei diesem
Verfahren wird zunächst die mögliche Existenz eines Texts in einem Bild \textit{entdeckt}. Alle Entdeckungen eines
Bildes werden der nächsten Stufe als Vorschläge präsentiert. Diese versucht dann, innerhalb der Vorschläge Text zu
\textit{erkennen}. Grundlage des Netzes ist eine Hierarchie mehrerer mit einander verbundener
\textit{convolutional layer}, an deren Ende ein \textit{fully connected layer} steht, welches die erkannten Buchstaben
ausgibt (\textit{aktiviert}) (vgl.~\cite{jaderberg2016}).

Shi et al.\ stellten 2017 ein Netzwerk vor, welches ebenfalls der Texterkennung in Fotografien dienen sollte. Wie
Jaderberg et al.\ setzen sie zunächst auf eine Hierarchie mehrerer verbundener \textit{convolutional layer}, fügen
jedoch vor der Buchstabenaktivierung \textit{long short-term memory layer} ein, deren Ausgabe später durch einen
handgeschriebenen Algorithmus dekodiert wird. (vgl.~\cite{shi2017})

\subsection{Ziel}
\label{einleitung:ziel}

Ziel dieser Arbeit ist der Entwurf eines neuronalen Netzes, welches Texte auf topographischen Karten zuverlässig
erkennen und extrahieren kann. Dieses Netz soll auf dem zur TU Dresden gehörigen HPC-System \textit{Taurus} zum Einsatz
kommen und die dort vorhandenen GPUs zur Beschleunigung verwenden.

Ferner soll untersucht werden, inwieweit ein bereits bestehendes \textit{trainiertes} Netzwerk, das für ein
anderes, aber verwandtes Problem (beispielsweise Texterkennung auf Straßenschildern) für die Texterkennung auf
topographischen Karten verwendet werden kann (\textit{Transfer Learning}).

\section{Daten und Methoden}
\label{daten}

\subsection{Reale und künstliche Daten}
\label{daten:daten}

Bevor ein neuronales Netz zum Einsatz gebracht werden kann, muss es mittels realer oder künstlich erzeugter Daten, die
dem realen Problem möglichst nahe kommen, \textit{trainiert} werden. Im vorliegenden Fall der topographischen Karten
wäre die Erzeugung einer ausreichend großen, das heißt dem \textit{Overfitting}
(vgl.\ Abschnitt~\ref{ergebnisse:overfitting}) vorbeugenden, realen Datenmenge sehr aufwendig, da dies das händische
Markieren und Ausschneiden tausender Beispiele erforderte. Es kommen daher für das \textit{Training} künstlich erzeugte
Daten zum Einsatz.

Innerhalb des ScaDS-Projektes lag zur Erzeugung dieser Daten bereits ein Generator vor (vgl.~\cite{schoelzel17}). Dieser
erzeugt auf leeren Kartenhintergründen mit topographischen Informationen (Bäume, Flüsse, Höhenlinien, etc.) zufällige
Buchstabenkombinationen in verschiedenen Schriftarten (siehe Abbildung~\ref{daten:daten:beispiele}).

\begin{figure}
    \centering
    \includegraphics[width = 0.9\linewidth]{img/Mabcsq.jpg}
    \caption{Beispiel für künstlich erzeugte Daten\label{daten:daten:beispiele}}
\end{figure}

\subsection{Bilderkennung}
\label{daten:bilderkennung}

Die mit dem Datengenerator erzeugten Bilder werden vor der Eingabe in das Netzwerk zunächst auf eine einheitliche
Größe (Breite: Wortlänge $L$ multipliziert mit 32, Höhe 64) skaliert. Anschließend werden sie in ein Graubild des
Typs \texttt{float32} umgewandelt und auf Werte zwischen 0 und 1 normalisiert.

Das Bilderkennungsverfahren folgt dem von \textit{Shi et al.}\ vorgestellten Ansatz (vgl.~\cite{shi2017}). Es wird eine
Hierarchie mehrerer miteinander verbundener \textit{convolutional layer}, \textit{max-pooling layer},
\textit{batch normalization layer} und schlussendlich eines bidirektionalen \gls{lstm} \textit{layers} erzeugt, an das
sich ein \textit{fully-connected layer} für die Buchstabenaktivierung anschließt (siehe
Tabelle~\ref{daten:bilderkennung:conv}).

Bis zu diesem Punkt entspricht das \textit{Trainings}-Netzwerk dem \textit{Inferenz}-Netzwerk.

\begin{table}
    \caption{Netzwerkkonfiguration des Bilderkennungsteils}
    \centering
    \begin{tabular}{|c|c|}
        \hline
        \textbf{Typ} & \textbf{Konfiguration}\\ \hline \hline
        Input & $(L * 32) * 64$ \textit{Grey-scale}-Bilder\\ \hline
        Convolution & 64 Filter, (3, 3)-Kernel, (1, 1)-Stride, same-Padding, relu-Aktivierung\\ \hline
        MaxPooling & (2, 2)-Fenster, (2, 2)-Stride\\ \hline
        Convolution & 128 Filter, (3, 3)-Kernel, (1, 1)-Stride, same-Padding, relu-Aktivierung\\ \hline
        MaxPooling & (2, 2)-Fenster, (2, 2)-Stride\\ \hline
        Convolution & 256 Filter, (3, 3)-Kernel, (1, 1)-Stride, same-Padding, relu-Aktivierung\\ \hline
        Convolution & 256 Filter, (3, 3)-Kernel, (1, 1)-Stride, same-Padding, relu-Aktivierung\\ \hline
        MaxPooling & (1, 2)-Fenster, (2, 2)-Stride\\ \hline
        Convolution & 512 Filter, (3, 3)-Kernel, (1, 1)-Stride, same-Padding, relu-Aktivierung\\ \hline
        BatchNormalization & - \\ \hline
        Convolution & 512 Filter, (3, 3)-Kernel, (1, 1)-Stride, same-Padding, relu-Aktivierung\\ \hline
        MaxPooling & (1, 2)-Fenster, (2, 2)-Stride\\ \hline
        Convolution & 512 Filter, (2, 2)-Kernel, (1, 1)-Stride, valid-Padding, relu-Aktivierung\\ \hline
        Reshape & - \\ \hline
        LSTM & Bidirektional, 512 Einheiten, alle Sequenzen, Vereinigung: Summe\\ \hline
        Dense & 53 Einheiten, he\_normal-Initialisierung, softmax-Aktivierung\\ \hline
    \end{tabular}
    \label{daten:bilderkennung:conv}
\end{table}

\subsection{Textklassifizierung}
\label{daten:textklassifizierung}

Im Unterschied zur Methode von \textit{Shi et al.}\ kommt für die Textklassifizierung bzw.\ -dekodierung kein spezieller
Algorithmus zum Einsatz, sondern das Verfahren des 2006 von \textit{Graves et al.}\ entwickelten \gls{ctc}.

Die \gls{ctc}-Methode wurde für die Erkennung nicht-segmentierter sequentieller Daten konzipiert. In dieser Form liegen
beispielsweise digitalisierte Handschriften, Sprach- oder Gestenaufnahmen vor, deren einzelne Sequenzschritte sich nicht
immer eindeutig voneinander abgrenzen lassen: Wann endet bei der Aussprache eines Wortes ein Buchstabe? Wann beginnt der
nächste? In Schreibschriften, die Wörter in einer einzigen Handbewegung produzieren, lassen sich Anfang und Ende
konkreter Buchstaben häufig ebenfalls nur schwierig bestimmen.

Zum Zeitpunkt der Vorstellung der \gls{ctc}-Methode war für diese Art von Aufgaben die Nutzung eines \gls{hmm} oder
\gls{crf} üblich, die jedoch erhebliches Vorwissen über das zu lösende Problem voraussetzten und die explizite
Definition von Abhängigkeiten verlangten (vgl.~\cite{graves2006}, S.~369).

Die Verwendung eines \gls{rnn} (wie etwa eines \gls{lstm}) zu diesem Zweck erfordert -- abgesehen von der Definition
der Ein- und Ausgangsformate -- kein weiteres Wissen über das Problem. \gls{rnn} eignen sich darüber hinaus gut für die
Modellierung von Sequenzen, da sie über einen internen Zustand verfügen. Die damals bekannten auf dem Gebiet des Machine
Learning verwendeten Zielfunktionen (Loss-Funktionen) mussten jedoch für jeden Punkt der Sequenz bzw.\ Zeitschritt
separat definiert werden. Dies hatte zur Folge, dass \gls{rnn} nur darauf trainiert werden konnten, voneinander gänzlich
unabhängige Klassifizierungen vorzunehmen. Die \textit{Trainings}-Daten mussten daher manuell segmentiert und die 
Ausgabe nachbearbeitet werden (vgl.~\cite{graves2006}, S.~369). 

Ein damals weit verbreiteter Ansatz bestand darin, \gls{rnn} mit \gls{hmm} zu kombinieren: \gls{hmm} können die
Sequenzen während des \textit{Trainings} automatisch segmentieren sowie die Ausgabe der \gls{rnn} verarbeiten. Dieses
\textit{hybride} System erbte jedoch die oben erwähnten Nachteile von \gls{hmm} (vgl.~\cite{graves2006}, S.~369 - 370).

Dagegen ermöglicht die \gls{ctc}-Methode die Zuordnung von \textit{Labels} zu nicht-segmentierten Eingabesequenzen.
Dadurch ist sie für Anwendungsfälle, die eine solche Segmentierung nicht brauchen (wie etwa der Erkennung von
Handschriften, bei der die Abgrenzung zwischen Buchstaben mitunter schwierig ist) besonders gut geeignet; jedoch ist sie
für Probleme, die eine solche Segmentierung voraussetzen (z.B.\ die Voraussage von Proteinstrukturen) weniger geeignet.
(vgl.~\cite{graves2006})

Im vorliegenden Problem der topographischen Karten wurde die \gls{ctc}-Methode vor allem deshalb ausgewählt, um dem
mitunter schreibschriftartigen Charakter historischer Schriftarten zu begegnen. \textit{Keras} selbst bietet keine
Implementierung der \gls{ctc}-Methode, jedoch ist sie über das \textit{TensorFlow}-Backend nutzbar.

Ab diesem Punkt besteht ein Unterschied zwischen dem \textit{Trainings}- und dem \textit{Inferenz}-Netzwerk. An den
in Abschnitt~\ref{daten:bilderkennung} beschriebenen Bildverarbeitungsteil schließt sich im \textit{Trainings}-Netzwerk
\textit{TensorFlows} \gls{ctc}-Loss-Funktion (\texttt{ctc\_batch\_cost}) an, die mittels eines \textit{Keras}-Lambdas
als eigene Schicht in das Netzwerk integriert werden kann; hier ersetzt sie die sonst über \textit{Keras} spezifizierte
Loss-Funktion (siehe Quelltext~\ref{daten:textklassifizierung:training}).

Bei der \textit{Inferenz} ist das Vorhandensein der \gls{ctc}-Loss-Funktion nicht notwendig. Stattdessen lässt sich die
Ausgabe der \gls{lstm}-Schicht mit \textit{TensorFlows} Funktion \texttt{ctc\_decode} direkt dekodieren (siehe
Quelltext~\ref{daten:textklassifizierung:inferenz}).

\begin{code}
\begin{minted}{python}
from keras import backend as K
from keras.layers import Input, Lambda
from keras.models import Model

def ctc_lambda_func(args):
    y_pred, labels, time_steps, label_length = args
    return K.ctc_batch_cost(labels, y_pred, time_steps,
                            label_length)

# Modelldefinition - Bilderkennungsanteil
# Dann:
labels       = Input(name = "labels", shape = [word_length],
                     dtype = "float32")
time_steps   = Input(name = "time_steps", shape = [1],
                     dtype = "int64")
label_length = Input(name "label_length", shape = [1],
                     dtype = "int64")

# prediction ist der letzte Dense-Layer der Bilderkennung
loss = Lambda(ctc_lambda_func, output_shape = (1, ), name = "CTC")
             ([prediction, labels, time_steps, label_length])

# CTC-Loss mit Bilderkennung kombinieren
model = Model(inputs = [input_data, labels,
                        time_steps, label_length],
              outputs = [loss])

# Loss wird in obigem Lambda berechnet -> hier nur Dummyfunktion
model.compile(loss = {"CTC": lambda y_true, y_pred: y_pred},
              optimizer = sgd, metrics = {...})

# Model kann jetzt für Training verwendet werden
\end{minted}
\captionof{listing}{Nutzung der CTC-Loss-Funktion beim \textit{Training}}
\label{daten:textklassifizierung:training}
\end{code}

\begin{code}
\begin{minted}{python}
import numpy as np

from keras import backend as K

alphabet = "ABCDEFGHIJKLMNOPQRSTUVWXYZabcdefghijklmnopqrstuvwxyz"

int_to_char = dict((i, c) for i, c in enumerate(alphabet))

def ctc_decode(y_preds):
    results = []
    y_pred_len = np.ones(y_preds.shape[0]) * y_preds.shape[1]

    # Dekodierung der LSTM-Ausgabe
    # greedy = True führt eine Best-Path-Suche durch,
    # andernfalls müsste vorher ein Wörterbuch definiert werden
    decoded, _ = K.ctc_decode(y_preds, input_length = y_pred_len,
                              greedy = True)

    # tf.keras.backend.ctc_decode gibt ein Tupel von
    # einelementigen Listen zurück. Das eine Element ist der
    # kodierte Buchstabe, die Vereinigung der Listen somit das
    # kodierte Wort.
    for label in K.get_value(decoded[0]):
        result = []
        for i in label:
            if i == -1:
                continue # überspringe blanks
            # Umwandlung von labels in Buchstaben
            result.append(int_to_char[i])
        results.append(''.join(result)) # dekodierter String
    return results

# Modelldefinition - Bilderkennungsanteil; Gewichte und Bilder
# laden. Dann:

# hier noch kodiert und mit Blanks
predictions = model.predict(images)

# hier dekodierter String
decodeds = ctc_decode(predictions)
\end{minted}
\captionof{listing}{Dekodierung im Zuge der \textit{Inferenz}}
\label{daten:textklassifizierung:inferenz}
\end{code}

\subsection{Umsetzung auf dem HPC-System \textit{Taurus}}
\label{daten:taurus}

\subsubsection{Verwendete Hardware}
\label{daten:taurus:hardware}

Sowohl das \textit{Training} als auch die \textit{Inferenz} wurden stets auf einem der im \textit{Taurus} vorhandenen
GPU-Knoten mit vier NVIDIA Tesla K80 und 62 GiB Arbeitsspeicher durchgeführt (Partition \texttt{gpu2}).

Der Datengenerator wurde (je nach Verfügbarkeit) auf einem der \textit{Taurus}-CPU-Knoten betrieben, in der Regel auf
einem Knoten der \texttt{sandy}-Partition mit 16 CPU-Kernen und 30\ GiB Arbeitsspeicher.

\subsubsection{Verwendete Module}
\label{daten:taurus:module}

Sowohl die \textit{Trainings}- und \textit{Inferenz}-Skripte als auch der Datengenerator wurden innerhalb der
SCS5-Umgebung verwendet.

Die Skripte für das \textit{Training} und die \textit{Inferenz} erfordern neben dem \textit{Keras}-Framework zusätzlich
die Bibliotheken \textit{NumPy} (für den Umgang mit großen Arrays) und \textit{OpenCV} (für die Bildvorverarbeitung).
Das Modulsystem des \textit{Taurus} stellt alle genannten Abhängigkeiten bereit; zusätzlich sorgt das Laden des
\textit{Keras}-Moduls dafür, dass die \textit{NumPy}-Bibliothek als Teil von Python automatisch mitgeladen wird. Für die
Ausführung genügt deshalb das Laden der \textit{Keras-} und \textit{OpenCV}-Module. Zum gegenwärtigen Zeitpunkt
(20.\ September 2018) werden \textit{Keras} in der Version \texttt{2.2.0-foss-2018a-Python-3.6.4} und \textit{OpenCV} in
der Version \texttt{3.4.1-foss-2018a-Python-3.6.4} geladen.

Der Datengenerator bringt diverse Abhängigkeiten mit sich, die durch das \textit{Taurus}-Modulsystem nicht vollständig
abgedeckt werden können. Aus diesem Grund wurde für seinen Einsatz im Rahmen des ScaDS-Projekts eine virtuelle
Python-Umgebung (\textit{venv}) angelegt, in der die benötigten Python-Pakete vorhanden sind. Diese \textit{venv} kam
auch bei der Anfertigung dieser Arbeit zum Einsatz. Sie befindet sich auf dem \textit{Taurus} im
ScaDS-Projektverzeichnis unter dem folgenden Pfad:

\texttt{/projects/p\_scads/keras/new\_venv2}

\section{Ergebnisse}

\subsection{Test- und Validierungsdaten}

\subsection{Test- und Validierungsergebnisse}

\subsection{Erkennungsraten}

\subsection{Performance}

\subsubsection{Nutzung mehrerer GPUs}

Der beabsichtigte Geschwindigkeitszuwachs beim \textit{Training} durch die Nutzung der für die Verwendung mehrerer GPUs
vorgesehenen \textit{Keras}-Befehle (wie in Abschnitt~\ref{daten:multigpu} beschrieben) fiel deutlich kleiner aus als
erwartet. So dauerte das \textit{Training} bei der Nutzung einer einzigen GPU 669 Sekunden bei einer Datensatzgröße
von 90123 Bildern (7 ms pro Schritt). Die Dauer der Verarbeitung des selben Datensatzes mit vier GPUs verkürzte sich auf
lediglich 515s (6 ms pro Schritt). Das entspricht einem Speedup $S$ von 1,299 sowie einer parallelen Effizienz $E$ von
0,325 und liegt somit weit unter den theoretisch erreichbaren Werten von $S = 4$ bzw. $E = 1$.

Wie existierende Benchmarks zeigen, ist dies kein inhärentes Problem von Deep-Learning-Netzwerken, der implementierten
Algorithmen oder des \textit{TensorFlow}-Backends (vgl.~\cite{tensorflowbench}). Es liegt deshalb nahe, das Problem bei
\textit{Keras} selbst zu suchen.

Tatsächlich zeigen Profiler-Ergebnisse, dass zwischen der Verarbeitung einzelner \textit{Batches} große Lücken
entstehen, während derer keine mit dem \textit{Training} zusammenhängenden Aufgaben durchgeführt werden.

HIER PROFILE UND BEGRÜNDUNG EINFÜGEN

Eine Diskussion dieses Problems in der Literatur konnte während der Recherche nicht festgestellt werden. Es existieren
jedoch erste Ansätze innerhalb der Keras-Nutzergemeinschaft, die unzureichende Asynchronität durch direkten Zugriff auf 
\textit{TensorFlow} zu umgehen (vgl.~\cite{zamecnik2017}).

\section{Fazit}

\subsection{Entwicklungsstand}

Es wurde gezeigt, dass Texterkennung in und -extraktion aus topographischen Karten mit der vorgestellten
Netzwerkarchitektur möglich ist und gute Erkennungsraten liefert. Vorbedingung dafür ist das Vorliegen einer genügend
großen Datenmenge für das \textit{Training}, da das Netzwerk ansonsten zu \textit{Overfitting} neigt.

Die Nutzung der Frameworks \textit{Keras} und \textit{TensorFlow} ermöglichte eine einfache Umsetzung für die Nutzung
der auf dem HPC-System \textit{Taurus} vorhandenen GPUs, auch wenn die Nutzung mehrerer GPUs zur Zeit noch nicht
optimal funktioniert.

\subsection{Ausblick}

Insbesondere die Gewinnung den historischen Schriftbildern entsprechender \textit{Trainings}-Daten gestaltete sich
schwierig. In seiner aktuellen Fassung behilft sich der Datengenerator mit aus dem vorliegenden Kartenmaterial
ausgeschnittenen Einzelbuchstaben, die dann aneinandergereiht werden; die Alternative besteht im mühevollen manuellen
Ausschneiden der ganzen Wörter, wie es für diese Arbeit für die in Abschnitt~\ref{} erwähnten Validierungsdaten getan
wurde.

Eine Verbesserung dieser Situation sowie daraus folgend der \textit{Inferenz}-Ergebnisse ließe sich vermutlich durch
den Einsatz der Schriftarten \textit{Kursivschrift} (vgl.~\cite{kursivschrift}) und \textit{Roemisch}
(vgl.~\cite{roemisch}) innerhalb des Datengenerators erreichen, da sie den historisch genutzten Schriftarten
weitestgehend entsprechen.

Ein weiterer Ansatz zur Verbesserung der Erkennungsrate könnte darin liegen, Vorder- und Hintergrund, das heißt Text
und Karte, voneinander zu trennen und getrennt auszuwerten.

Hinsichtlich der Performance im Hinblick auf die GPU-Nutzung steht eine genauere Untersuchung der effizienten Nutzung
mehrerer GPUs sowie der Beschleunigung durch eine künftige \gls{ctc}-Implementierung für GPUs aus.

Zu untersuchen wäre ferner, inwieweit sich ein Netzwerk, das bereits auf ein ähnlich gelagertes Problem angelernt
wurde (wie etwa die Texterkennung in Fotografien), im Rahmen von \textit{Transfer Learning} für die Texterkennung in
topographischen Karten verwenden lässt.


\listoflistings

%\include{anhang}

\end{document}
