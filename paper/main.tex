\documentclass[utf8,german,analyse,lot,lof,lol]{zihpub}

\usepackage{caption}
\usepackage{datetime}
\usepackage{etoolbox}
\usepackage[acronyms,nonumberlist,nopostdot,toc,savewrites,xindy]{glossaries}
\usepackage[section,newfloat]{minted}
\usepackage[super]{nth}
\usepackage{pgfplots}
\usepackage{siunitx}
\usepackage{tcolorbox}

\makeglossaries

\newacronym{ocr}{OCR}{\textit{optical character recognition}}


\newenvironment{code}{\captionsetup{type=listing}}{}
\SetupFloatingEnvironment{listing}{name=Quelltext}

\BeforeBeginEnvironment{minted}{\begin{tcolorbox}}
\AfterEndEnvironment{minted}{\end{tcolorbox}}

\newcommand{\lstfont}[1]{\color{#1}\small\ttfamily}

\definecolor{keyword-green}{RGB}{0, 128, 0}

\author{Jan Stephan}
\title{Texterkennung in topographischen Karten: Untersuchungen mit dem Deep-Learning-Framework Keras in einer
       HPC-Systemumgebung}
\matno{3755136}
\betreuer{Dr.\ Peter Winkler}
\date{28. September 2018}
% \copyrighterklaerung{Die nachfolgend aufgeführten Wortmarken sind Markenzeichen ihrer jeweiligen Inhaber, auch wenn sie
%                     im Text dieser Arbeit nicht als solche gekennzeichnet sind. CUDA{\textregistered},
%                     Quadro{\textregistered} und GeForce{\textregistered} sind eingetragene Markenzeichen der Firma
%                     NVIDIA. DirectX{\textregistered} ist ein eingetragenes Markenzeichen der Firma Microsoft.
%                     OpenGL{\texttrademark} und OpenCL{\texttrademark} sind nichteingetragene Markenzeichen der
%                     Khronos Group. Xeon Phi{\texttrademark} ist ein nichteingetragenes Markenzeichen der Firma Intel.}
\acknowledgments{Für die tatkräftige Hilfe bei der manuellen Extraktion der Realdaten aus dem Kartenmaterial bedanke ich
                 mich herzlich bei N. Werner. TODO: Vornamen erfragen}

\bibfiles{bibliography.bib}
\def\UrlBreaks{\do\/\do-}

% Germanisiere LaTeX
\deftranslation[to=German]{Acronyms}{Abkürzungsverzeichnis}
\deftranslation[to=German]{Glossary}{Glossar}
\renewcommand*{\listlistingname}{Quelltextverzeichnis}

\fancyhf{}
\fancyhead[L]{Jan Stephan -- Texterkennung in topographischen Karten}

\begin{document}

\glsaddall

\printglossary
\printglossary[type=\acronymtype]
%\printglossary[type=nomencl,style=nomenclstyle]

\section{Einleitung}
\label{einleitung}

\subsection{Motivation}
\label{einleitung:motivation}

Die Analyse historischer topographischer Karten ist eine bedeutende Aufgabe der Raum- und Landschaftsplanung, um die
historische Entwicklung von Siedlungen und Naturräumen nachvollziehen zu können. Besondere Aufmerksamkeit kommt dabei
der automatisierten Texterkennung zu, um digitalisiertes Kartenmaterial schnell katalogisieren, sortieren oder
durchsuchen zu können.

Die kombinierte Texterkennung in und -extraktion aus Bildern ist ein Standardproblem der Bildanalyse sowohl im Bereich
des klassischen \textit{Computer Vision} (d.h. Objekterkennung durch spezielle Algorithmen) als auch des seit einigen
Jahren populärer gewordenen Gebiet des maschinellen Lernens (siehe auch Abschnitt~\ref{einleitung:forschung}). Dieser
\gls{ocr} genannte Prozess kann in zahlreichen Anwendungsdomänen eingesetzt werden, wie etwa der automatisierten
Erkennung von Kennzeichen, der Digitalisierung von Büchern oder der Informationsextraktion aus nicht digitalisierten
Verträgen oder Business-Dokumenten.

Alle beispielhaft aufgeführten Fälle haben jedoch gemeinsam, dass die Texterkennung aufgrund des starken Kontrastes
zwischen Text und Hintergrund und des Mangels an störenden Artefakten relativ einfach möglich ist. Die Anwendung auf
historische topographische Karten gestaltet sich aufgrund einiger Faktoren vergleichsweise schwierig. So ist das
Farbspektrum trotz der Fülle an Informationen in der Regel auf die Farbe des (vergilbten) Papiers und eine einzige
Tintenfarbe eingeschränkt. Das führt dazu, dass der zu extrahierende Text sich farblich nicht von weiteren Markierungen
auf der Karte, wie etwa Straßen, Bäumen oder Höhenzügen, unterscheiden lässt und somit zahlreiche störende Artefakte
bei der Erkennung ignoriert werden müssen. Darüber hinaus ist die Linienstärke der Buchstaben historischer Schriftarten
stellenweise sehr dünn, sodass dickere Linien des Bildhintergrundes weitaus dominanter erscheinen als der eigentlich
vordergründige Text.

Diese Schwierigkeiten machen die Texterkennung in historischen topographischen Karten zu einer besonders
herausfordernden und interessanten Aufgabe im Bereich des maschinellen Lernens.

\subsection{Forschungsstand}
\label{einleitung:forschung}

Die Idee der maschinellen \gls{ocr} beschäftigt Ingenieure, Informatiker und Erfinder seit mehr als einem Jahrhundert.
Bereits 1914 stellte Edmund Fournier d'Albe ein \textit{Optophon} vor, das nach Erkennung eines Buchstabens einen
bestimmten Ton abspielte und auf diese Weise Blinden das Lesen ermöglichen sollte (vgl.~\cite{albe1914}). Wenige Jahre
später meldete der in Dresden bei der Firma \textit{Zeiss Ikon} tätige Emanuel Goldberg ein Patent für die von ihm
entwickelte \glqq statistische Maschine\grqq\ an, mit deren Hilfe die in Form von Buchstaben- und Ziffernkombinationen
vorliegenden Metadaten auf Filmrollen durchsucht werden konnten (vgl.~\cite{goldberg1931}).

In jüngerer Zeit ist die \gls{ocr} im Bereich des maschinellen Lernens häufiger untersucht worden. Eines der
bekanntesten Projekte in diesem Bereich ist die zunächst von der Firma \textit{Hewlett-Packard}, dann von der Firma
\textit{Google} entwickelte \textit{Tesseract OCR Engine} (vgl.~\cite{smith2007}), die mittlerweile neben
algorithmusbasierter Texterkennung den Einsatz eines neuronalen Netzes ermöglicht (vgl.~\cite{tesseract40}).

Jaderberg et al.\ präsentierten 2016 ein mehrstufiges Verfahren für die Texterkennung in Fotografien. Bei diesem
Verfahren wird zunächst die mögliche Existenz eines Texts in einem Bild \textit{entdeckt}. Alle Entdeckungen eines
Bildes werden der nächsten Stufe als Vorschläge präsentiert. Diese versucht dann, innerhalb der Vorschläge Text zu
\textit{erkennen}. Grundlage des Netzes ist eine Hierarchie mehrerer mit einander verbundener
\textit{convolutional layer}, an deren Ende ein \textit{fully connected layer} steht, welches die erkannten Buchstaben
ausgibt (\textit{aktiviert}) (vgl.~\cite{jaderberg2016}).

Shi et al.\ stellten 2017 ein Netzwerk vor, welches ebenfalls der Texterkennung in Fotografien dienen sollte. Wie
Jaderberg et al.\ setzen sie zunächst auf eine Hierarchie mehrerer verbundener \textit{convolutional layer}, fügen
jedoch vor der Buchstabenaktivierung \textit{long short-term memory layer} ein, deren Ausgabe später durch einen
handgeschriebenen Algorithmus dekodiert wird. (vgl.~\cite{shi2017})

\subsection{Ziel}
\label{einleitung:ziel}

Ziel dieser Arbeit ist der Entwurf eines neuronalen Netzes, welches Texte auf topographischen Karten zuverlässig
erkennen und extrahieren kann. Dieses Netz soll auf dem zur TU Dresden gehörigen HPC-System \textit{Taurus} zum Einsatz
kommen und die dort vorhandenen GPUs zur Beschleunigung verwenden.

Ferner soll untersucht werden, inwieweit ein bereits bestehendes \textit{trainiertes} Netzwerk, das für ein
anderes, aber verwandtes Problem (beispielsweise Texterkennung auf Straßenschildern) für die Texterkennung auf
topographischen Karten verwendet werden kann (\textit{Transfer Learning}).

\section{Daten und Methoden}
\label{daten}

\subsection{Reale und künstliche Daten}
\label{daten:daten}


\subsection{Bilderkennungsverfahren}
\label{daten:bilderkennung}

\subsection{\textit{Transfer Learning}}
\label{daten:transfer}

\subsection{Netzbeschreibung}
\label{daten:netzbeschreibung}

\subsection{Umsetzung auf dem HPC-System \textit{Taurus}}
\label{daten:taurus}

Das im Abschnitt~\ref{daten:netzbeschreibung} beschriebene Netz ist auf dem zur TU Dresden gehörigen HPC-System
\textit{Taurus} ohne Einschränkungen lauffähig. Zur Ausführung der Python-Skripte für \textit{Training} und
\textit{Inferring} ist lediglich das Laden einiger im Folgenden genannten Module nötig. Die Nutzung mehrerer GPUs
erfordert kleinere Anpassungen am Quelltext der Skripte, die ebenfalls in diesem Abschnitt aufgeführt werden.

\subsubsection{Verwendete Hardware}
\label{daten:taurus:hardware}

Sowohl das \textit{Training} als auch das \textit{Inferring} wurden stets auf einem der im \textit{Taurus} vorhandenen
GPU-Knoten mit vier NVIDIA Tesla K80 und 62 GiB Arbeitsspeicher durchgeführt (Partition \texttt{gpu2}).

Der Datengenerator wurde (je nach Verfügbarkeit) auf einem der \textit{Taurus}-CPU-Knoten betrieben, in der Regel auf
einem Knoten der \texttt{sandy}-Partition mit 16 CPU-Kernen und 30\ GiB Arbeitsspeicher.

\subsubsection{Verwendete Module}
\label{daten:taurus:module}

Sowohl die \textit{Trainings}- und \textit{Inferring}-Skripte als auch der Datengenerator wurden innerhalb der
SCS5-Umgebung verwendet.

Die Skripte für das \textit{Training} und das \textit{Inferring} erfordern neben dem \textit{Keras}-Framework zusätzlich
die Bibliotheken \textit{NumPy} (für den Umgang mit großen Arrays) und \textit{OpenCV} (für die Bildvorverarbeitung).
Das Modulsystem des \textit{Taurus} stellt alle genannten Abhängigkeiten bereit; zusätzlich sorgt das Laden des
\textit{Keras}-Moduls dafür, dass die \textit{NumPy}-Bibliothek als Teil von Python automatisch mitgeladen wird. Für die
Ausführung genügt deshalb das Laden der \textit{Keras-} und \textit{OpenCV}-Module. Zum gegenwärtigen Zeitpunkt
(20.\ September 2018) werden \textit{Keras} in der Version \texttt{2.2.0-foss-2018a-Python-3.6.4} und \textit{OpenCV} in
der Version \texttt{3.4.1-foss-2018a-Python-3.6.4} geladen.

Der Datengenerator bringt diverse Abhängigkeiten mit sich, die durch das \textit{Taurus}-Modulsystem nicht vollständig
abgedeckt werden können. Aus diesem Grund wurde für seinen Einsatz im Rahmen des ScaDS-Projekts eine virtuelle
Python-Umgebung (\textit{venv}) angelegt, in der die benötigten Python-Pakete vorhanden sind. Diese \textit{venv} kam
auch bei der Anfertigung dieser Arbeit zum Einsatz. Sie befindet sich auf dem \textit{Taurus} im
ScaDS-Projektverzeichnis unter dem folgenden Pfad:

\texttt{/projects/p\_scads/keras/new\_venv2}

\subsubsection{Nutzung mehrerer GPUs}
\label{daten:taurus:multigpu}

Die Dauer des \textit{Trainings} kann -- abhängig von der konkreten Netzwerkkonfiguration und der Datenmenge -- mitunter
sehr viel Zeit in Anspruch nehmen, bis hin zu mehreren Tagen. Es liegt deshalb nahe, den benötigten Zeitaufwand durch
die Erhöhung des Parallelisierungsgrades zu verringern.

Das \textit{Keras}-Framework unterstützt in Verbindung mit dem \textit{TensorFlow}-Backend die parallele Nutzung
mehrerer GPUs. Dazu wird das Netz zuerst im Arbeitsspeicher der CPU angelegt und dann auf die vorhandenen GPUs verteilt.
Das so entstandene \texttt{multi\_gpu\_model} kann in der Folge wie ein normales \textit{Keras}-Modell verwendet werden
(siehe Quelltext~\ref{daten:taurus:multigpu:source}).

\begin{code}
\begin{minted}{python}
import tensorflow as tf
# ...

from keras.utils.multi_gpu_utils import multi_gpu_model
# ...

# Keras-Beschränkung: gpu_num muss ein Vielfaches von 2 sein
if (gpu_num >= 2) and (num_gpu % 2 == 0):
    with tf.device('/cpu'):
        model = Model(inputs = [input_data, labels,
                                input_length, label_length],
                      outputs = [loss_out])
    model = multi_gpu_model(model, gpu_num)
else:
    # Standardfall: gpu_num <= 1 oder ungerade
    model = Model(inputs = [input_data, labels,
                            input_length, label_length],
                  outputs = [loss_out])

model.compile(...)
model.fit(...)
\end{minted}
\captionof{listing}{Nutzung mehrerer GPUs mit \textit{Keras}}
\label{daten:taurus:multigpu:source}
\end{code}

\section{Ergebnisse}
\label{ergebnisse}

\subsection{Erfolgskriterien}
\label{ergebnisse:erfolg}

Für die Erfolgsbewertung der \textit{Inferenz} werden in den folgenden Abschnitten zwei \textit{Scores} gebildet: der
\textit{Wordscore} und der \textit{Charscore}. Der \textit{Wordscore} ist ein binärer Wert und gibt an, ob ein ganzes
Wort korrekt erkannt wurde. Der \textit{Charscore} wird durch eine Zählung der korrekt erkannten Buchstaben eines
Wortes ermittelt und ist ein Wert zwischen 0 und 1, wobei 1 die korrekte Erkennung aller Buchstaben an ihrem erwarteten
Platz bedeutet. Anders ausgedrückt bedeutet ein \textit{Wordscore} von 1 immer einen \textit{Charscore} von 1; ein
\textit{Charscore} kleiner 1 impliziert immer einen \textit{Wordscore} von 0.

Beispiele:

\begin{itemize}
    \item Der Text \glqq Bramel\grqq\ wird korrekt als \glqq Bramel\grqq\ erkannt. \textit{Wordscore} =
          \textit{Charscore} = 1
      \item Der Text \glqq Ta\textbf{\color{red}n}nen\grqq\ wird als \glqq Ta\textbf{\color{red}u}nen\grqq\ erkannt.
          \textit{Wordscore} = 0, \textit{Charscore} = 0.8333
\end{itemize}

Dieses Verfahren erlaubt eine grobe statistische Auswertung des Lernerfolgs, berücksichtigt jedoch einige Sonderfälle
nicht.

Beispiel:

\begin{itemize}
    \item Der Text \glqq Wehdel\grqq\ wird als \glqq NWehdel\grqq\ erkannt. \textit{Wordscore} = \textit{Charscore} = 0
\end{itemize}

Obwohl der Text \glqq Wehdel\grqq\ in Gänze richtig erkannt wurde, sorgt das falsch erkannte \glqq N\grqq\ am
Wortbeginn für einen \textit{Wordscore} von 0. Da der \textit{Charscore} nur Buchstaben an ihrem korrekten Platz
berücksichtigt, diese jedoch durch das \glqq N\grqq\ um eine Position nach rechts verschoben wurden, ist er in diesem
Beispiel ebenfalls 0.

\subsection{\textit{Overfitting}}
\label{ergebnisse:overfitting}

Ein generelles Problem des maschinellen Lernens im Allgemeinen sowie der \gls{ctc}-Methode im Besonderen
(vgl.~\cite{graves2006}, S.\ 376) ist die Frage, wie eine zu große Abhängigkeit des Netzwerks von den
\textit{Trainings}-Daten vermieden werden kann. Diese \textit{Overfitting} genannte Sichtverengung des Netzwerks sorgt
zwar für exzellente Ergebnisse, wenn die \textit{Inferenz} auf den \textit{Trainings}-Daten durchgeführt wird,
verschlechtert jedoch die Erkennungsrate für unbekannte Daten.

Das im Rahmen dieser Arbeit entwickelte Netzwerk ist vor allem bei der Verwendung kleiner \textit{Trainings}-Datensätze
von \textit{Overfitting} betroffen, wie die in der Tabelle~\ref{} aufgeführten Messergebnisse zeigen. Im weiteren
Verlauf wurde daher nur mit Datensatzgrößen jenseits von 100.000 Bildern gearbeitet (siehe
Abschnitt~\ref{ergebnisse:daten}).

TODO: Ergebnisse für kleine Datensätze hier einfügen, wenn SLURM wieder funktioniert

\subsection{Trainings- und Validierungsdaten}
\label{ergebnisse:daten}

Für das \textit{Training} wurden insgesamt sechs verschiedene Datensätze verwendet, wovon jeweils drei mit dem Präfix
\texttt{wl6} und drei mit \texttt{wl28} versehen sind. Die \texttt{wl6}-Datensätze werden für das Training eines
Netzwerks für Texte mit fixer Wortlänge (sechs Buchstaben) verwendet und enthalten Bilder zufälliger Größe. Die
\texttt{wl28}-Datensätze sind für das Training eines Netzwerks erzeugt worden, welches Wörter mit beliebiger Länge
zwischen 2 und 8 Buchstaben erkennt. Die Bilder dieser Datensätze haben eine feste Größe von $256 \times 64$ Pixeln
(vgl.\ Tabelle~\ref{ergebnisse:daten:training}).

Die Validierung erfolgte durch \textit{Inferenz} auf generierten und realen, aus dem vorliegenden Kartenmaterial
ausgeschnittenen Daten. Diese Datensätze wurden nicht für das \textit{Training} verwendet; ihr Benennungsschema
entspricht dem der \textit{Trainings}-Daten (vgl.\ Tabelle~\ref{ergebnisse:daten:validierung}).

\begin{table}
    \caption{Trainingsdatensätze}
    \centering
    \begin{tabular}{|c|c|c|c|c|}
        \hline
        \textbf{Name} & \textbf{Bildanzahl} & \textbf{Textlänge} & \textbf{Bildgröße} & \textbf{Datengröße}\\ \hline \hline
        \texttt{wl6\_120k} & 120.164 & 6 & variabel & \SI{2,3357}{\gibi\byte} \\ \hline
        \texttt{wl6\_250k} & 250.000 & 6 & variabel & \SI{4,8841}{\gibi\byte} \\ \hline
        \texttt{wl6\_500k} & 500.000 & 6 & variabel & \SI{9,7756}{\gibi\byte} \\ \hline
        \texttt{wl28\_120k} & TODO & variabel & $256 \times 64$ & \SI{0}{\gibi\byte} \\ \hline
        \texttt{wl28\_250k} & TODO & variabel & $256 \times 64$ & \SI{0}{\gibi\byte} \\ \hline
        \texttt{wl28\_250k} & TODO & variabel & $256 \times 64$ & \SI{0}{\gibi\byte} \\ \hline
    \end{tabular}
    \label{ergebnisse:daten:training}
\end{table}

\begin{table}
    \caption{Validierungsdatensätze}
    \centering
    \begin{tabular}{|c|c|c|c|c|c|}
        \hline
        \textbf{Name} & \textbf{Typ} & \textbf{Bildanzahl} & \textbf{Textlänge} & \textbf{Bildgröße} & \textbf{Datengröße}\\ \hline \hline
        \texttt{wl6\_1000} & generiert & 1000 & 6 & variabel & \SI{19,4173}{\mebi\byte} \\ \hline
        \texttt{wl6\_real} & real & 27 & 6 & variabel & \SI{549,5}{\kibi\byte} \\ \hline
        \texttt{wl28\_1000} & generiert & 1000 & variabel & $256 \times 64$ & \SI{0}{\mebi\byte} \\ \hline
        \texttt{wl28\_real} & real & TODO & variabel & $256 \times 64$ & \SI{0}{\kibi\byte} \\ \hline
    \end{tabular}
    \label{ergebnisse:daten:validierung}
\end{table}

\subsection{\textit{Scores}, \textit{Losses} und Genauigkeit}
\label{ergebnisse:scores}

(siehe Tabelle~\ref{ergebnisse:scores:scores})

\begin{table}
    \caption{Scores}
    \centering
    \begin{tabular}{|c|c|c|c|}
        \hline
        \textbf{Trainingsdatensatz} & \textbf{Validierungsdatensatz} & $\sum$ \textbf{Wordscore} / Gesamtzahl & \textbf{Charscore} ($\varnothing$)\\ \hline \hline
        \texttt{wl6\_120k} & \texttt{wl6\_120k} & TODO & TODO \\ \hline
        \texttt{wl6\_120k} & \texttt{wl6\_1000} & TODO & TODO \\ \hline
        \texttt{wl6\_120k} & \texttt{wl6\_real} & TODO & TODO \\ \hline
        \texttt{wl6\_250k} & \texttt{wl6\_250k} & \num{226432/250000} & \num{0,9086} \\ \hline 
        \texttt{wl6\_250k} & \texttt{wl6\_1000} & \num{759/1000} & \num{0,863} \\ \hline 
        \texttt{wl6\_250k} & \texttt{wl6\_real} & \num{4/27} & \num{0,4753} \\ \hline 
        \texttt{wl6\_500k} & \texttt{wl6\_500k} & TODO & TODO \\ \hline 
        \texttt{wl6\_500k} & \texttt{wl6\_1000} & TODO & TODO \\ \hline 
        \texttt{wl6\_500k} & \texttt{wl6\_real} & TODO & TODO \\ \hline \hline
        \texttt{wl28\_120k} & \texttt{wl28\_120k} & TODO & TODO \\ \hline
        \texttt{wl28\_120k} & \texttt{wl28\_1000} & TODO & TODO \\ \hline
        \texttt{wl28\_120k} & \texttt{wl28\_real} & TODO & TODO \\ \hline
        \texttt{wl28\_250k} & \texttt{wl28\_250k} & TODO & TODO \\ \hline
        \texttt{wl28\_250k} & \texttt{wl28\_1000} & TODO & TODO \\ \hline
        \texttt{wl28\_250k} & \texttt{wl28\_real} & TODO & TODO \\ \hline
        \texttt{wl28\_500k} & \texttt{wl28\_500k} & TODO & TODO \\ \hline
        \texttt{wl28\_500k} & \texttt{wl28\_1000} & TODO & TODO \\ \hline
        \texttt{wl28\_500k} & \texttt{wl28\_real} & TODO & TODO \\ \hline
    \end{tabular}
    \label{ergebnisse:scores:scores}
\end{table}

\subsection{Performance}

\subsubsection{\textit{Training}}

\subsubsection{\textit{Inferenz}}

\subsubsection{Nutzung mehrerer GPUs}

Der beabsichtigte Geschwindigkeitszuwachs beim \textit{Training} durch die Nutzung der für die Verwendung mehrerer GPUs
vorgesehenen \textit{Keras}-Befehle (wie in Abschnitt~\ref{daten:taurus:multigpu} beschrieben) fiel deutlich kleiner aus
als erwartet. So dauerte das \textit{Training} bei der Nutzung einer einzigen GPU \SI{669}{\second} bei einer
Datensatzgröße von 90123 Bildern (\SI{7}{\milli\second} pro Schritt). Die Dauer der Verarbeitung des selben Datensatzes
mit vier GPUs verkürzte sich auf lediglich \SI{515}{\second} (\SI{6}{\milli\second} pro Schritt). Das entspricht einem
Speedup $S$ von \num{1,299} sowie einer parallelen Effizienz $E$ von \num{0,325} und liegt somit weit unter den
theoretisch erreichbaren Werten von $S = 4$ bzw. $E = 1$.

Wie existierende Benchmarks zeigen, ist dies kein inhärentes Problem von Deep-Learning-Netzwerken, der implementierten
Algorithmen oder des \textit{TensorFlow}-Backends (vgl.~\cite{tensorflowbench}). Wahrscheinlich sind daher zwei
Ursachen ausschlaggebend:

\begin{enumerate}
    \item \textit{Keras'} \texttt{multi\_gpu\_model} könnte laut einiger Hinweise der Nutzergemeinschaft nicht sehr gut
          implementiert sein - unter anderem wird mangelnde Überlappung zwischen Speichertransfers und Berechnungen auf
          der GPU beklagt. Dieses Problem soll sich durch direkten Zugriff auf die \textit{TensorFlow}-API umgehen
          lassen (vgl.~\cite{zamecnik2017}).
    \item \textit{TensorFlows} \gls{ctc}-Implementierung kann in Ermangelung einer entsprechenden Implementierung nicht
          von der Beschleunigung durch GPUs profitieren; eine zeitnahe Umsetzung ist derzeit (26.\ September 2018) nicht
          in Sicht (vgl.~\cite{hibbert2016}).
\end{enumerate}

Der Einfluss dieser Punkte auf Speedup und parallele Effizienz bedürfen zukünftig weiterer Untersuchung.

\section{Fazit}

\subsection{Entwicklungsstand}

Es wurde gezeigt, dass Texterkennung in und -extraktion aus topographischen Karten mit der vorgestellten
Netzwerkarchitektur möglich ist und gute Erkennungsraten liefert. Vorbedingung dafür ist das Vorliegen einer genügend
großen Datenmenge für das \textit{Training}, da das Netzwerk ansonsten zu \textit{Overfitting} neigt. Künstliche
\textit{Trainings}-Daten sind daher für ein funktionierendes Netzwerk essentiell.

Die Nutzung der Frameworks \textit{Keras} und \textit{TensorFlow} ermöglichte eine einfache Umsetzung für die Nutzung
der auf dem HPC-System \textit{Taurus} vorhandenen GPUs.

\subsection{Ausblick}

\subsubsection{Qualität der Trainingsdaten und Erkennungsraten}

Insbesondere die Gewinnung den historischen Schriftbildern entsprechender \textit{Trainings}-Daten gestaltete sich
schwierig. In seiner aktuellen Fassung behilft sich der Datengenerator mit aus dem vorliegenden Kartenmaterial
ausgeschnittenen Einzelbuchstaben, die dann aneinandergereiht werden; die Alternative besteht im mühevollen manuellen
Ausschneiden der ganzen Wörter, wie es für diese Arbeit für die in Abschnitt~\ref{ergebnisse:daten} erwähnten
Validierungsdaten getan wurde.

Eine Verbesserung dieser Situation sowie daraus folgend der \textit{Inferenz}-Ergebnisse ließe sich vermutlich durch
den Einsatz der Schriftarten \textit{Kursivschrift} (vgl.~\cite{kursivschrift}) und \textit{Roemisch}
(vgl.~\cite{roemisch}) innerhalb des Datengenerators erreichen, da sie den historisch genutzten Schriftarten
weitestgehend entsprechen.

Ein weiterer Ansatz zur Verbesserung der Erkennungsrate könnte darin liegen, Vorder- und Hintergrund, das heißt Text
und Karte, voneinander zu trennen und getrennt auszuwerten.

Eine genauere Untersuchung erfordert das \textit{Training} mit Datensätzen, die größer als \num{250000} Bilder sind. So
zeigte sich bei der Verwendung eines Datensatzes mit \num{500000} Bildern ein massiver Einbruch der Erkennungsrate,
der noch unter der Rate eines Netzwerks, das mit \num{120000} Bildern trainiert wurde, lag. Dieses Verhalten konnte
bisher noch nicht erklärt werden.

\subsubsection{Variable Wortlängen}

Das im Rahmen dieser Arbeit entwickelte Netzwerk ist zwar auf Texte mit einer fixen Wortlänge beschränkt, eignet sich
prinzipiell aber auch für die Erkennung von Texten mit variabler Wortlänge. Hierzu bedarf es einer Untersuchung der
folgenden Punkte:

\begin{itemize}
    \item Da das Netzwerk auf \gls{lstm}-Schichten basiert, ist eine vorherige Festsetzung der Dimensionen der
          eingehenden Bilder erforderlich. In der Variante mit fixer Wortlänge ergibt sich die Breite eines Bildes
          aus der Anzahl der vorhandenen Buchstaben (siehe Abschnitt~\ref{daten:bilderkennung}). Dagegen muss bei der
          variablen Wortlänge eine gemeinsame Bildbreite für alle möglichen Wortlängen gefunden werden.
    \item Ausgehend von dem vorherigen Punkt stellt sich die Frage, in welchem Format die künstlichen
          \textit{Trainings}-Daten vorliegen sollen. Ein erster Ansatz, den Generator ausschließlich Bilder mit fester
          Breite und Höhe, aber Texten mit verschiedener Schriftgröße generieren zu lassen und das Netzwerk auf diesen
          Daten zu traineren, lieferte sehr schlechte Ergebnisse ($C < 0,2$ für den Trainingsdatensatz und
          $C \approx 0,01$ für einen Validierungsdatensatz mit realen Daten). Eine Alternative könnte sein, das Netzwerk
          ausschließlich mit künstlichen Daten der selben Wortlänge (z.B.\ einer vorher festgelegten Maximallänge) zu
          trainieren. Dieses Netzwerk lässt man dann Worte verschiedener Breite erkennen. Ein erster Versuch mit einem
          Netzwerk, das auf eine Wortlänge von 6 Buchstaben trainiert wurde, zeigte vielversprechende Ergebnisse für
          \textbf{reale} Daten mit Wortlängen zwischen 2 und 8 Buchstaben ($C \approx 0,25$). Möglicherweise ließe sich
          auch ein iteratives Lernverfahren umsetzen (das Netzwerk lernt erst Wörter der Länge 2, dann der Länge 3,
          usw.\ bis zum Maximum).
\end{itemize}

\subsubsection{Performance-Verbesserungen}

Hinsichtlich des Laufzeitverhaltens steht eine genauere Untersuchung der effizienten Nutzung mehrerer GPUs aus. Eine
solche Erhöhung des Parallelisierungsgrades ist aufgrund der Dauer des \textit{Trainings} wünschenswert und wird von
\textit{Keras} und \textit{TensorFlow} prinzipiell unterstützt (\textit{Keras} bietet hier das
\texttt{multi\_gpu\_model}-API). Erste Testläufe dazu haben bereits stattgefunden, es sind jedoch für einen effizienten
Multi-GPU-Ansatz weitere Arbeiten notwendig.

\subsubsection{Weitere Arbeiten}

Zu untersuchen wäre ferner, inwieweit sich ein Netzwerk, das bereits auf ein ähnlich gelagertes Problem angelernt
wurde (wie etwa die Texterkennung in Fotografien), im Rahmen von \textit{Transfer Learning} für die Texterkennung in
topographischen Karten verwenden lässt.

Schließlich wäre eine verbesserte \textit{Score}-Berechnung wünschenswert, die auch Sonderfälle wie die in
Abschnitt~\ref{ergebnisse:erfolg} genannten berücksichtigen kann, beispielsweise mittels einer Substring-Suche.



\listoflistings

%\include{anhang}

\end{document}
