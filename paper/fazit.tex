\section{Fazit}

\subsection{Entwicklungsstand}

Es wurde gezeigt, dass Texterkennung in und -extraktion aus topographischen Karten mit der vorgestellten
Netzwerkarchitektur möglich ist und gute Erkennungsraten liefert. Vorbedingung dafür ist das Vorliegen einer genügend
großen Datenmenge für das \textit{Training}, da das Netzwerk ansonsten zu \textit{Overfitting} neigt.

Die Nutzung der Frameworks \textit{Keras} und \textit{TensorFlow} ermöglichte eine einfache Umsetzung für die Nutzung
der auf dem HPC-System \textit{Taurus} vorhandenen GPUs, auch wenn die Nutzung mehrerer GPUs zur Zeit noch nicht
optimal funktioniert.

\subsection{Ausblick}

Insbesondere die Gewinnung den historischen Schriftbildern entsprechender \textit{Trainings}-Daten gestaltete sich
schwierig. In seiner aktuellen Fassung behilft sich der Datengenerator mit aus dem vorliegenden Kartenmaterial
ausgeschnittenen Einzelbuchstaben, die dann aneinandergereiht werden; die Alternative besteht im mühevollen manuellen
Ausschneiden der ganzen Wörter, wie es für diese Arbeit für die in Abschnitt~\ref{} erwähnten Validierungsdaten getan
wurde.

Eine Verbesserung dieser Situation sowie daraus folgend der \textit{Inferenz}-Ergebnisse ließe sich vermutlich durch
den Einsatz der Schriftarten \textit{Kursivschrift} (vgl.~\cite{kursivschrift}) und \textit{Roemisch}
(vgl.~\cite{roemisch}) innerhalb des Datengenerators erreichen, da sie den historisch genutzten Schriftarten
weitestgehend entsprechen.

Ein weiterer Ansatz zur Verbesserung der Erkennungsrate könnte darin liegen, Vorder- und Hintergrund, das heißt Text
und Karte, voneinander zu trennen und getrennt auszuwerten.

Hinsichtlich der Performance im Hinblick auf die GPU-Nutzung steht eine genauere Untersuchung der effizienten Nutzung
mehrerer GPUs sowie der Beschleunigung durch eine künftige \gls{ctc}-Implementierung für GPUs aus.

Zu untersuchen wäre ferner, inwieweit sich ein Netzwerk, das bereits auf ein ähnlich gelagertes Problem angelernt
wurde (wie etwa die Texterkennung in Fotografien), im Rahmen von \textit{Transfer Learning} für die Texterkennung in
topographischen Karten verwenden lässt.
