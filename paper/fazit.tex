\section{Fazit}

\subsection{Entwicklungsstand}

\subsection{Ausblick}

Eine Verbesserung der Ergebnisse ließe sich vermutlich durch eine Verbesserung der \textit{Trainings}-Daten erreichen.
Insbesondere der Einsatz der Schriftarten \textit{Kursivschrift} und \textit{Roemisch} innerhalb des Datengenerators
könnte hier messbare Fortschritte erzielen.

Ein weiterer Ansatz zur Verbesserung der Erkennungsrate könnte darin liegen, Vorder- und Hintergrund, das heißt Text
und Karte, voneinander zu trennen und getrennt auszuwerten.

Hinsichtlich der Performance im Hinblick auf die GPU-Nutzung steht eine genauere Untersuchung der effizienten Nutzung
mehrerer GPUs sowie der Beschleunigung durch eine künftige \gls{ctc}-Implementierung für GPUs aus.

Zu untersuchen wäre ferner, inwieweit sich ein Netzwerk, das bereits auf ein ähnlich gelagertes Problem angelernt
wurde (wie etwa die Texterkennung in Fotografien), im Rahmen von \textit{Transfer Learning} für die Texterkennung in
topographischen Karten verwenden lässt.
